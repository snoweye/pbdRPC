
{\color{red} \bf Disclaimer:}\\
The findings and conclusions in this article have not been
formally disseminated by the U.S. Department of Health \& Human Services
nor by the U.S. Department of Energy,
and should not be construed to represent any determination or
policy of University, Agency, Administration and National Laboratory.

{\color{red} \bf Warning:}\\
This document is written to explain the main
functions of \pkg{pbdRPC}~\citep{Chen2017pbdRPCpackage}, version 0.1-1.
Every effort will be made to ensure that future versions are consistent with
these instructions, but features in later versions may not be explained
in this document.

Information about the functionality of this package,
and any changes in future versions can be found on website:
``Programming with Big Data in R'' (pbdR) at
\url{http://r-pbd.org/}~\citep{pbdR2012}.




\section[Introduction]{Introduction}
\label{sec:introduction}
\addcontentsline{toc}{section}{\thesection. Introduction}

This package, \pkg{pbdRPC}~\citep{Chen2017pbdRPCpackage},
provides one high-level function, \code{srpc()}, that can securely send
commands to remote servers via \code{ssh} (OpenSSH) or
\code{plink}/\code{plink.exe} (PuTTY).
The high-level function is a very light yet secure implementation because
the communications are encrypted, by default SSH version 2 using RSA key pairs,
between a local client and remote servers.

The high-level function \code{srpc()} is a unified interface via system call
to either \code{ssh} or \code{plink} across most popular OSs
including Linux, Mac OSX, MS Windows, and Solaris.
Simply speaking, the function \code{srpc()} is a wrapper
of two low-level functions, \code{ssh()} and \code{plink()}.
However, these functions can ask remote servers to execute commands without
logging in the servers provided that an authentication is setup properly.

Four RPC controls are provided by the package to simply the functions:
\begin{enumerate}
\item \code{.pbd_env$RPC.CT} is main RPC controls taking care several basic
      functionalities of three functions, \code{srpc()}, \code{ssh()}, and
      \code{plink()}.
\item \code{.pbd_env$RPC.LI} has information of login account for logging in
      the remote server include authentication using private keys.
      See Section~\ref{sec:handling_login_information} for details.
\item \code{.pbd_env$RPC.RR} has examples of executing multiple commands
      on a remote server which is an application related
      to an \proglang{R} package, \pkg{remoter}~\citep{remoter}. See
      Section~\ref{sec:application_rr} for details.
\item \code{.pbd_env$RPC.CS} has examples of executing multiple commands
      on a \pkg{pbdCS} cluster which is an application related
      to an \proglang{R} package, \pkg{pbdCS}~\citep{pbdCS}. See
      Section~\ref{sec:application_cs} for details.
\end{enumerate}
Note that \code{.pbd_env} will be first generated when the library \pkg{pbdRPC}
is loaded, then default objects \code{RPC.CT}, \code{RPC.LI}, \code{RPC.RR},
and \code{RPC.CS} will be generated.

In general, only \code{RPC.CT} and \code{RPC.LI} are required to be 
set by users according to their environment configurations.
The \code{RPC.RR} and \code{RPC.Cs} are simple templates to show two
pbdR applications. Users are welcomed to adopt those templates and to
develop personal applications wherever automations can be benefit
by remote procedure calls.

Most OSs (Linux, Solaris, Mac OSX) have the system command \code{ssh} (OpenSSH)
installed, so the
\code{ssh()} is a wrapper function to the system \code{ssh} command.
For Windows, the \code{plink.exe} (from PuTTY) will be compiled with
\pkg{pbdRPC}, so the \code{plink()} is a wrapper function to the
executable file, \code{plink.exe}.
Note that for non-Windows system, the \code{plink} can be compiled as well.


\subsection[Basic \code{ssh} and \code{srpc()}]{Basic \code{ssh} and \code{srpc()}}
\label{sec:basic_rpc}
\addcontentsline{toc}{subsection}{\thesubsection. Basic \code{ssh}\  and \code{srpc()}}

Suppose a \code{sshd} is set and running correctly on a server running a Linux
system at an ip address ``192.168.56.101'' and a port ``22''.
Further, suppose an account
called ``snoweye'' is created and a password for the account is set on
the server.

From a terminal or a shell of non-Windows systems, one may use
\begin{Command}[title=Basic \code{ssh} in shell]
$ ssh snoweye@192.168.56.101 'whoami'
\end{Command}
to access the server and to ask the server to execute a command \code{whoami}.
Typing the password for the login account may be needed.
The command \code{whoami} is available on most Linux systems, it should
return the command result, ``snoweye'', on the screen/stdout without
logging in a shell environment on the server.
In the same setup, the command \code{whoami} can be replaced
by any other proper programs, shell scripts, or procedures.
For Windows system, one may use \code{plink.exe} instead of \code{ssh}
from a terminal \code{cmd.exe}. See Section~\ref{sec:basic_plink} for details.

Within \proglang{R}, the example below will have the
same results as the above shell command.
\begin{Code}[title=Basic \code{srpc()} in \pkg{pbdRPC} and \proglang{R}]
> library(pbdRPC, quietly = TRUE)
>
> ### Alter login information as needed
> rpcopt_set(user = "snoweye", hostname = "192.168.56.101")
> srpc("whoami")
\end{Code}
The command results may be captured by \proglang{R} as well.

Regardless the system,
the high level function \code{srpc()} can unify the calls to
either \code{ssh()} or \code{plink()} functions.
One may use \code{ssh()} in non-Windows system, but
use \code{plink()} in Windows system.
The \code{srpc()} automatically
detects the system first, then calls the corresponding function.
Currently, no external \code{plink.exe} or \code{plink} is implemented
even though it is possible.
The details of \code{ssh()} and \code{plink()} are given in examples below.


\subsection[Basic \code{ssh()}]{Basic \code{ssh()}}
\label{sec:basic_ssh}
\addcontentsline{toc}{subsection}{\thesubsection. Basic \code{ssh()}}

Inside \proglang{R} and via \pkg{pbdRPC}, this can be done by
\begin{Code}[title=Basic \code{ssh()} in \pkg{pbdRPC} and \proglang{R}]
> library(pbdRPC, quietly = TRUE)
> ssh("snoweye@192.168.56.101 'whoami'")
\end{Code}
provided all other options (port, forwarding, etc) are set correctly.
Note that the password for the account may be required when an
authentication file (\code{id_rda}) is not available.

Note that multiple commands can be automatically given at once
as shell commands under an shell prompt, such as ``\code{;}'', ``\code{\&\&}'',
``\code{>}'', ``\code{<}'', ``\code{|}'' or ``\code{\&}'' etc.
For example, the code below will tell current id, date/time, and files.
\begin{Command}[title=Multiple commands to \code{ssh} in shell]
$ ssh snoweye@192.168.56.101 'whoami;date;ls -a'
\end{Command}
The multiple commands can be applied to \code{ssh()} and \code{plink()} as
\begin{Code}[title=Multipel commands to \code{ssh()} in \pkg{pbdRPC} and \proglang{R}]
> library(pbdRPC, quietly = TRUE)
> ssh("snoweye@192.168.56.101 'whoami;date;ls -a'")
\end{Code}
See Section~\ref{sec:application_rr} and \code{.pbd_env$RPC.RR} for
more details.


\subsection[Basic \code{plink.exe} and \code{plink()}]{Basic \code{plink.exe} and \code{plink()}}
\label{sec:basic_plink}
\addcontentsline{toc}{subsection}{\thesubsection. Basic \code{plink.exe}\  and \code{plink()}}

In Windows system and inside the \code{cmd.exe}, one may similarly use
the code below
\begin{Command}[title=Basic \code{plink.exe} in \code{cmd.exe}]
C:\> plink.exe snoweye@192.168.56.101 'whoami'
\end{Command}
to access the server provided \code{plink.exe} is in the \code{PATH}.

Inside \code{RGui} and via \pkg{pbdRPC}, this can be done by
\begin{Code}[title=Basic \code{plink()} in \pkg{pbdRPC} and \proglang{R}]
> library(pbdRPC, quietly = TRUE)
> plink("snoweye@192.168.56.101 'whoami'")
\end{Code}
provided all other options (port, forwarding, etc) are set correctly.
The multiple commands can be applied to \code{plink()} as well.

By default,
the \code{plink()} will open an \code{cmd.exe} to execute the
command \code{whoami} because the password input is not allowed inside
\code{RGui}.
When the authentication file (\code{id_rsa.ppk}) is available,
one may want to disable the opening \code{cmd.exe} below.
\begin{Code}[title=Advance \code{plink()} in \pkg{pbdRPC} and \proglang{R}]
> .pbd_env$RPC.CT$use.shell.exec <- FALSE
> ret <- plink("snoweye@192.168.56.101 'whoami'")
> print(ret)
\end{Code}
Because the \code{shell.exec()} is disable, the \code{plink()} call
may accept returns of the remote server and capture/save the returns
in an \proglang{R} object, \code{ret}.

The \code{use.shell.exec} is for Windows system only and
required to be \code{TRUE} when \code{RGui} is mainly used. The
\code{plink()} in \code{RGui} may hang forever or crash when input/typing of a
password or a passphrase is needed for logging in the server.
\code{RGui} has different stdin and stdout than a usual terminal.
The \code{use.shell.exec} can be switched to \code{FALSE} when
the authentication is correct and no passphrase is needed, i.e.
{\color{red} no stdin input/typing. }
However, \code{Rcmd} running within a \code{cmd.exe} may be OK with stdin
input/typing when \code{use.shell.exec = FALSE}.

Other solutions to replace internal \code{plink.exe} of \pkg{pbdRPC}
include:
\begin{itemize}
\item
The \code{plink.exe} can be installed from the PuTTY as well.
\item
Windows PowerShell and git also provide \code{ssh.exe} but
additional installation/configuration is unavoidable.
\end{itemize}

