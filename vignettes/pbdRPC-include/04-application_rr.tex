
\section[An Application with \pkg{remoter}]{An Application with \pkg{remoter}}
\label{sec:application_rr}
\addcontentsline{toc}{section}{\thesection. An Application with \pkg{remoter}}

The \pkg{remoter}~\citep{remoter} and \pkg{pbdZMQ}~\citep{Chen2015pbdZMQpackage}
provide client/server interface to control a remote \proglang{R}
(e.g. running on a single server, Xubuntu, ip=192.168.56.101)
from a local \proglang{R} (e.g. running on a single laptop, Windows 8).
Combining with \pkg{pbdMPI}~\citep{Chen2012pbdMPIpackage} and
\pkg{pbdCS}~\citep{pbdCS}, one may extent the
remote \proglang{R} to the \proglang{R} clusters by running
\proglang{R}'s in a distributed/SPMD environment.
\begin{itemize}
\item See \citet{xsede16} for an introduction of \pkg{remoter} and \pkg{pbdCS}.
\item See \url{http://github.com/snoweye/user2016.demo}
      for a demo of both packages.
\item See pbdR-Tech (\url{http://snoweye.github.io/pbdr/}) and
      HPSC (\url{http://snoweye.github.io/hpsc/}) websites
      for more applications of SPMD and
      how to utilize \proglang{R} in clusters~\citep{hpsc2012}.
\end{itemize}

In a simplified scenario such as the setting in
Section~\ref{sec:handling_login_information}, one may use the following
commands to ``start'', ``check'', and ``kill'' a remote \proglang{R} server
under a shell environment provided \code{Rscript} is in \code{PATH}
of the login server (pre-load or set by the \code{00_set_devel_R}).
\begin{Command}[title=\pkg{remoter} server at 192.168.56.101]
$ source ~/work-my/00_set_devel_R
$ nohup Rscript -e 'remoter::server()' > .rrlog 2>&1 < /dev/null &
$ ps ax|grep '[r]emoter::server'
$ kill -9 $(ps ax|grep '[r]emoter::server'|awk '{print $1}')
\end{Command}

In an well established server, one can use \code{ssh} or \code{plink.exe}
to send those commands from a local laptop.
Furthermore, one may also use \pkg{pbdRPC} directly within an \proglang{R}
environment to send those commands. The example is in next.
\begin{Code}[title=Using \pkg{pbdRPC} to control \pkg{remoter}]
> library(pbdRPC, quietly = TRUE)
>
> ### Alter login information as needed
> rpcopt_set(user = "snoweye", hostname = "192.168.56.101")
> .pbd_env$RPC.CT$use.shell.exec <- FALSE
>
> preload <- "source ~/work-my/00_set_devel_R; "
> start_rr(preload = preload)
character(0)
>
> library(remoter)
Loading required package: pbdZMQ

Attaching package: 'remoter'

The following object is masked from 'package:grDevices':

    dev.off

The following objects are masked from 'package:utils':

    ?, help

> client(addr = "192.168.56.101")
WARNING: server not secure; communications are not encrypted.

remoter> 1+1
[1] 2 
remoter> q()
>
> check_rr()
[1] " 2014 ?        Sl     0:00 /home/snoweye/work-my/local/R-devel/lib64/R/bin/exec/R --slave --no-restore -e remoter::server()"
> kill_rr()
character(0)
\end{Code}
where \code{client()} is for connect to the remote \proglang{R} server
started by \code{start_rr()}.
Note that all commands in the above example were typed inside a
local \proglang{R} in the local laptop. However, the computation
\code{1+1} was done by a remote \proglang{R} on the server (192.168.56.101).

The \code{start_rr()}, \code{check_rr()}, and \code{kill_rr()} are
all wrapper functions of \code{rpc()} to submit different commands stored in
\code{.pbd_env$RPC.RR$start}, \code{.pbd_env$RPC.RR$check}, and
\code{.pbd_env$RPC.RR$kill}, respectively.
The tedious details of \code{RPC.RR} are in next which all can be
simply sent by \code{rpc()} to execute on the server.
\begin{Code}[title=\code{RPC.RR} for controlling \pkg{remoter}]
> .pbd_env$RPC.RR
$check
[1] "ps ax|grep '[r]emoter::server'"

$kill
[1] "kill -9 $(ps ax|grep '[r]emoter::server'|awk '{print $1}')"

$start
[1] "nohup Rscript -e 'remoter::server()' > .rrlog 2>&1 < /dev/null &"

$preload
[1] "source ~/work-my/00_set_devel_R; "

\end{Code}

