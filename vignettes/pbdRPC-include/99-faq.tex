

\section[FAQs]{FAQs}
\label{sec:faqs}
\addcontentsline{toc}{section}{\thesection. FAQs}


\subsection[General]{General}
\label{sec:general}
\addcontentsline{toc}{subsection}{\thesubsection. General}

\begin{enumerate}
\item {\bf\color{blue} Q:}
      Does \pkg{pbdRPC} support Windows system? \\
      {\bf\color{blue} A:}
      Yes, the \code{plink.exe} from PuTTY will be the program to send
      commands to remote servers. An internal built \code{plink.exe} will
      be provided and wrapped by the \pkg{pbdRPC} command \code{plink()}.

\item {\bf\color{blue} Q:}
      Is an authentication used in \pkg{pbdRPC}? How does it work? \\
      {\bf\color{blue} A:}
      Yes, the authentication is the same way to \code{ssh} and
      \code{plink.exe} provided public and private keys are setup correctly.
      For example, when an RSA key is used, the \code{ssh} will by
      default search
      \code{~/.ssh/id_rsa} or via the option ``\code{-i ./id_rsa}'' for a local
      private key.
      Similarly,
      the \code{plink.exe} uses the option ``\code{-i ./id_rsa.ppk}'' for a
      local private key.
      Inside \pkg{pbdRPC}, one can use the options of the control
      \code{.pbd_env$RPC.LI$priv.key} and \code{.pbd_env$RPC.LI$pri.key.ppk} to
      indicate the file of the private key. Then, \code{ssh()}, \code{plink()},
      and \code{rpc()} commands will automatically access those files,
      accordingly.

\item {\bf\color{blue} Q:}
      Can a \code{ssh} private key be converted to \code{plink}'s private
      key? i.e. convert OpenSSH format to PuTTY format. \\
      {\bf\color{blue} A:}
      Yes, the \code{puttygen} on linux can convert the \code{id_rsa}
      (OpenSSH format) to \code{id_rsa.ppk} (PuTTY format) as in next.
\begin{Command}
$ sudo apt-get install putty
$ puttygen id_rsa -O private -o id_rsa.ppk
\end{Command}

\end{enumerate}
